\chapter{Use Cases}
\label{usecases}

\section{Use Case Diagramm}

\section{Use Case Beschreibungen}
\label{usecase:description}

\subsection{UC 001: Register}
\label{uc:001-register}
\begin{tabular}{|l|p{0.6\textwidth}|}
\hline
\textbf{Use Case Name} 	&	Register	\\ \hline
\textbf{Primary Actor} 	&		\\ \hline
\textbf{Stakeholders}	&		\\ \hline
\textbf{Preconditions}	&		\\ \hline
\textbf{Main Success Scenario}	& 	
\begin{enumerate}
	\item Der EnterpriseManager gibt eine valide Emailadresse und zweimal ein sicheres Passwort ein.
\end{enumerate}
\\ \hline
\textbf{Extensions}	&
\\ \hline
\end{tabular}

\subsection{UC 002: Login}
\label{uc:002-login}


\subsection{UC 003: Logout}
\label{uc:003-logout}

\subsection{UC 004: Manage Presentation}
\label{uc:004-manage-pres}

\begin{tabular}{|l|p{0.6\textwidth}|}
\hline
\textbf{Use Case Name} 	&	Manage Presentation	\\ \hline
\textbf{Primary Actor} 	&		\\ \hline
\textbf{Stakeholders}	&		\\ \hline
\textbf{Preconditions}	&	Anzahl bestehender Präsentationen ist kleiner gleich max_presentations des Abo-Plans.	\\ \hline
\textbf{Main Success Scenario}	&
\begin{enumerate}
	\item Ein eingeloggter User kann Präsentationen hinzufügen
	\item Eine Präsentation besteht aus Slides in einer definierten Reihenfolge
	\item Eine Präsentation ist in der Grösse durch max_slides_pp des Abo-Plans beschränkt.
\end{enumerate}
\\ \hline
\textbf{Extensions}	& 	\\ \hline
\end{tabular}

\subsection{UC 005: Manage Slide}
\label{uc:005-manage-slide}

\begin{tabular}{|l|p{0.6\textwidth}|}
\hline
\textbf{Use Case Name} 	&	Manage Slide	\\ \hline
\textbf{Primary Actor} 	&		\\ \hline
\textbf{Stakeholders}	&		\\ \hline
\textbf{Preconditions}	&	Anzahl Slides in einer Präsentation ist kleiner gleich max_slides_pp des Abo-Plans.	\\ \hline
\textbf{Main Success Scenario}	&
\begin{enumerate}
	\item Ein eingeloggter User kann einer Präsentation Slides hinzufügen
	\item Ein Slide kann in einer anderen Präsentation wiederverwendet werden (shared = true)
	\item Ein Slide kann mithilfe eines WYSIWYG-Editors benutzerfreundlich bearbeitet werden
	\item Slides können in der Reihenfolge via Drag'n'Drop angepasst werden
\end{enumerate}
\\ \hline
\textbf{Extensions}	& 	\\ \hline
\end{tabular}

\subsection{UC 006: Share Presentation}
\label{uc:006-share-pres}

\subsection{UC 007: View Presentation}
\label{uc:007-view-pres}

\subsection{UC 008: Print Presentation}
\label{uc:008-print-pres}

\subsection{UC 009: Manage Comment}
\label{uc:009-manage-comment}

\subsection{UC 010: Manage Codebase}
\label{uc:010-manage-codebase}


\subsection{UC 011: Manage Code Fragment}
\label{uc:011-manage-code-frag}


\subsection{UC 012: Update External Resources}
\label{uc:012-update-external-res}

\begin{tabular}{|l|p{0.6\textwidth}|}
\hline
\textbf{Use Case Name} 	&	Update External Resources	\\ \hline
\textbf{Primary Actor} 	&		\\ \hline
\textbf{Stakeholders}	&		\\ \hline
\textbf{Preconditions}	&		\\ \hline
\textbf{Main Success Scenario}	&
\begin{enumerate}
	\item Die Präsentation wird angezeigt
\end{enumerate}
\\ \hline
\textbf{Extensions}	& 	\\ \hline
\end{tabular}

\subsection{UC 013: Manage Channel}
\label{uc:013-manage-channel}

\begin{tabular}{|l|p{0.6\textwidth}|}
\hline
\textbf{Use Case Name} 	&	Manage Channel	\\ \hline
\textbf{Primary Actor} 	&	User	\\ \hline
\textbf{Stakeholders}	&	-	\\ \hline
\textbf{Preconditions}	&	Der Abo-Plan des Users hat enable_channels auf true.	\\ \hline
\textbf{Main Success Scenario}	&
\begin{enumerate}
	\item Ein angemeldeter User kann Channels erstellen
	\item In diesen können Präsentationen gruppiert und geordnet werden.
\end{enumerate}
\\ \hline
\textbf{Extensions}	& 	\\ \hline
\end{tabular}

\subsection{UC 014: Subscribe Channel}
\label{uc:014-subscribe-channel}

\subsection{UC 015: Bookmark Presentation}
\label{uc:015-bookmark-pres}

\subsection{UC 016: Manage Profile}
\label{uc:016-manage-profile}