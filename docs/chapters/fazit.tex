\chapter{Fazit}
\label{chap:fazit}

\section{Gelerntes}
Wir konnten uns in der kurzen Zeit viel Wissen über aktuellste Frontend- und Backend-Technologien aneignen. Dies erlaubte die Entwicklung einer sehr weit ausgebauten API. Das Konzept wurde grundsätzlich getestet und die Anforderung für ein fertiges Produkt wurden definiert. Durch das Projekt sind grundlegende Entscheide bereits gefällt und getestet worden. 

Die Absprachen im Team waren zu beginn des Projekts sehr gut verlaufen. Beide Teammitglieder wussten genau, was sie wie umzusetzen haben. So konnten die Anforderungen nach Routen bei Frontend direkt vom anderen Teammitglied im Backend umgesetzt werden. Auch die Absprachen mit dem Dozenten verliefen sehr konstruktiv. Gegen Ende des Projekts war jedoch spürbar, dass die Konzentration nachgelassen hatte. Leider wurde der Kontakt zum Betreuer nicht mehr intensiv gepflegt, was wiederum zu Missverständnissen führte innerhalb des Teams. Positiv hervorzuheben ist aber der Einsatzwille, welcher sich auch in den 4 durchgeführten Hackathons (Samstags oder Sonntags) äusserte, welche während des Projekts durchgeführt wurden. Durch die Anwesenheit von beiden konnte das Teamwork gefördert werden und Absprachen konnten direkt mündlich geführt werden. Leider konnte diese Mentalität nicht bis zum Schluss weitergezogen werden, auch wegen der Abwesenheit von Herrn Schmid.

In künftigen Projekten muss der Kommunikation intern und extern mehr Bedeutung beigemessen werden. Hier müssten im Projektplaungstool (es wurde \emph{trello} verwendet) nicht nur Tasks für Code und Doku sondern auch Kommunikationsaufgaben erfasst werden.

\section{Ausblick}
Eine künfige Thesis (oder auch Projekt 1/2) kann auf der geleisteten Arbeit basieren. Das Datenbank-Konzept und die Kommunikation zwischen Frontend und Backend sind erprobt. Die eingesetzten Technologien haben sich bewährt. Es ist jedoch anzumerken, dass alle Technologien eine gewisse Einarbeitungszeit haben. Für ein Proof-of-Concept waren die eingesetzten Frameworks unter Umständen bereits zu massiv. Das Konzept bietet viele Erweitungsmöglichkeiten. Es konnten im Rahmen der Projektarbeit auch noch nicht alle Funktionalitäten vollständig umgesetzt werden. Eine weiterführende Arbeit kann hier ansetzen und die erstellten Dokumentationen und Codelines verwenden.