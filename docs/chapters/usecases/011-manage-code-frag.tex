\subsection{UC 011: Manage Code Fragment}
\label{uc:011-manage-code-frag}

\begin{tabular}{|l|p{0.6\textwidth}|}
\hline
\textbf{Use Case Name} 	&	Manage Code Fragment	\\ \hline
\textbf{Primary Actor} 	&		\\ \hline
\textbf{Stakeholders}	&		\\ \hline
\textbf{Preconditions}	&	Eine Codebase existiert	\\ \hline
\textbf{Main Success Scenario}	&
\begin{enumerate}
	\item Bei Erstellen eines Slides kann Code direkt aus einer Codebase angezeigt werden
	\item Dazu muss im WYSIWYG-Editor der entsprechende Button gewählt werden. Folgende Felder müssen ausgefüllt werden:
	\begin{enumerate}
	    \item Filepath
	    \item Revision\#
	    \item Branch
	    \item Start [Line]
	    \item Ende [Line]
	\end{enumerate}
	\item Das System generiert so im Hintergrund einen Shortcode, der auch im Source-Modus des Editors manuell eingegeben werden kann
\end{enumerate}
\\ \hline
\textbf{Extensions}	& 	\\ \hline
\end{tabular}